\section{Introduction}\label{sec:introduction}


Modern research papers often use highly technical language, domain-specific terms,
and dense methodological detail, making them difficult for non-experts and casual readers to understand.
The long format and conceptual complexity pose a barrier to broader communication,
especially for people outside the academic community who may still be curious about recent advances.

Traditionally, turning a research paper into a comic involves significant manual effort:
one must first read and comprehend the paper, then craft a narrative with appropriate dialogue,
and finally illustrate or generate the visuals to match the story.
With the rise of diffusion-model-as-a-tool multimodal pipelines, where LLMs serve as high-level planners and tools like
Stable Diffusion~\cite{rombach2022high} or DALL·E2~\cite{ramesh2021zero} generate the images,
this process can be partially automated.
However, these pipelines face critical limitations:
1) It is difficult to precisely control visual elements like character positioning, text rendering, or facial expressions.
2) Maintaining character consistency across different panels or scenes is notoriously unreliable.
3) Small but meaningful visual errors (e.g., wrong diagram, mismatched labels) are hard to fix.

With the emergence of unified multimodal foundation models,
like Chameleon~\cite{team2024chameleon}, TransFusion~\cite{zhou2024transfusion}, and MAR~\cite{li2024autoregressive},
that operate on a unified architecture for both text and image inputs/outputs, we are now seeing more promising results.
These models allow for fine-grained image control and significantly improve character consistency across scenes.
However, they still have limitations: they may produce less artistic variation or visual texture
compared to pure diffusion models, and they sometimes struggle with creative or stylistic rendering.

In this project, we aim to build a semi-automated pipeline that converts research papers into comics that are:
Story-rich, engaging for casual readers;
Visually consistent across scenes;
Understandable without prior domain knowledge;
Generated quickly, without requiring the operator to deeply understand the source paper.

Our method involves prompting an LLM to read and analyze the paper, extract key ideas and results,
and convert them into a narrative with well-structured scenes and panel-by-panel descriptions.
Characters are designed to reflect the structure of the paper,
and image generation is handled using both traditional tool-based systems and newer autoregressive multimodal models.
We further compare results between the two approaches and explore techniques to improve consistency, realism, and visual quality.


